\documentclass{hw}
\usepackage{xcolor}

% STUDENTS: Please fill these out with your information
\newcommand{\name}{Matthew Shen}
\newcommand{\netid}{mds377}
\newcommand{\collaborators}{Linjing Rao(lr534)}
% / STUDENT

\newcommand{\hwnum}{1}
\newcommand{\duedate}{September 26, 11:59pm ET}
\renewcommand{\title}{Stable Matching, Greedy, Divide and Conquer}

\newcommand{\submission}{\textbf{Submission.}}

\newtheorem{claim}{Claim}

\begin{document}

%%%%%%%%%%%%%%%%%%%%%%%%%%%%%%%%%%%%%%%%%%%%%%%%%%%%%%%%%%%%%%%%%%%%%%%%%%%%%%%%
% Problem 1: G-S algorithm

\begin{problem}
Your friend is wrapping up medical school
and applying for residency programs, but is concerned and confused about how the matching
system works. As the local expert on algorithms, your colleague wants your 
help understanding how matchings work. 

\begin{subproblem}
  (\textit{2 points})
  Your friend tried to implement the textbook Gale-Shapley (G-S)
  algorithm in Python. We provide this code in the homework materials in \texttt{problem\_1/p1\_a.py}. 
  Using this implementation, your friend thinks they found a way to propose a
  ranking that unfairly advantages them in getting the school of their choice.

  There is a logical bug in the implementation. Provide a minimal test case
  demonstrating the bug, i.e., an input with the smallest possible $n$ 
  that, when run with the buggy implementation, outputs a non-stable matching. Write your test case input in \texttt{problem\_1/p1a\_test.txt}.  
  
\end{subproblem}

\begin{subproblem}
  (\textit{8 points})
  Now we turn to characterizing the performance of
  this implementation. Fix the bug from part (a) and conduct an 
   of the implementation (see homework instructions
  in the first page). Include the fixed code in \texttt{problem\_1/p1\_b.py}.

  Provide a brief explanation of the performance you observe, and explain why the implementation does not achieve $\bigO(n^2)$ run time advertised for the G-S algorithm.
\end{subproblem}

\begin{solution}
The environment used for the empirical performance analysis can be seen below:
    \begin{itemize}
        \item CPU: Intel(R) Core(TM) i7-8500Y CPU @ 1.50GHz (4 threads, 4.20GHz)
        \item OS \& Version: Debian GNU/Linux 11 (bullseye)
        \item Memory: 15.53 GB
    \end{itemize}
    
    The graph created from the empirical performance analysis can be seen below:
    \begin{figure}[h]
      \centering
          \includegraphics[width=0.5\textwidth]{figures/problem-1b.png}
    \end{figure}



    The following code provided does not reach $\bigO(n^2)$. For the worst case scenario all hosptials have the same preference order for all doctors, and all doctors have the same preference order for all hospital. Additionally, the hospitals list is in the same order as the preferences that all the doctors have; therefore, a stable match is when the hosptial number and docter number matches (i.e. \{ 0:0, ... , n:n\}).This means, in the first loop through the first hosptial gets matched and all other hospitals are not matched. In the second loop the program goes through all the hospitals again except for the first/matched hosptial. In the second loop on the second hosptial gets matched because all other hosptials want to get the second student, but the second student's top choice is the second hosptial. This means we get a loop has a time complexity of $\bigO(n)$ and is run $\bigO(n!)$ times. This means the algorithm is $\bigO(n!)$.  



\end{solution}

\begin{subproblem}
\newcommand{\worstrank}{{\tt wr}}
(\textit{10 points})
Correct and improve the run time of the G-S implementation and turn it in for
  auto-grading. It must be correct (always outputting a stable matching) and should run in the expected $\bigO(n^2)$ time. 
  Provide a description of the optimizations you implemented.
  Additionally, provide an empirical performance analysis of your
  implementation in the same environment (same system and configuration) that
  you performed the earlier analysis. Include your new implementation in \texttt{problem\_1/p1\_c.py}.
\end{subproblem}

\begin{solution}
Please answer here.
\end{solution}

\end{problem}


\newpage
%%%%%%%%%%%%%%%%%%%%%%%%%%%%%%%%%%%%%%%%%%%%%%%%%%%%%%%%%%%%%%%%%%%%%%%%%%%%%%%%
% Problem 2

\begin{problem}
In the {\em interval covering problem},
one is given a time interval $[0,M]$
and a collection of closed subintervals
$\mathcal{I} = \{[a_i,b_i] \mid i=1,2,\ldots,n\}$
whose union is $[0,M]$. (Interpret intervals as jobs
and, interpret $a_i$ and $b_i$ as the {\em start time}
and {\em finish time} of job $[a_i,b_i]$.)
The goal is to find a subcollection
$\mathcal{J} \subseteq \mathcal{I}$, containing
as few intervals as possible, such that
the union of the intervals in $\mathcal{J}$
is still equal to $[0,M]$.

You can assume that the
numbers $M,a_1,a_2,\ldots,a_n,b_1,b_2,\ldots,b_n$
are all non-negative integers, and that
$a_i < b_i$ for all $i$. You should assume
that the list of intervals given in the
problem input could be arbitrarily ordered;
in other words, don't make an assumption that
the input to the algorithm presents the intervals
sorted according to any particular criterion.

(\textit{20 points}) Below we have listed four greedy algorithms for
this problem. At least one of them is correct,
and at least one is incorrect. For {\em every} incorrect
algorithm, provide an example of an input instance
such that the algorithm outputs an incorrect answer
in your write-up.
(Either a set of intervals that fails to cover $[0,M]$,
or a set that covers $[0,M]$ but has a greater number
of intervals than necessary.) For {\em every} correct
algorithm, write, ``The algorithm is correct.''
For {\em at least one} of the correct algorithms,
implement the algorithm to run in $O(n \log n)$; that is 
$\Theta(n \log n)$ or faster.
In the write-up, indicate which algorithm you implemented
and briefly describe your implementation strategy. Include your implementation in \texttt{problem\_2/p2.py}.

  \begin{subproblem}
    \textbf{Select Latest Finish Time:}
  Initialize $\mathcal{J}=\emptyset$.
  Until the intervals in $\mathcal{J}$ cover all of $[0,M]$,
  repeat the following loop: find the maximum $c \in [0,M]$
  such that $[0,c]$ is covered by the intervals in $\mathcal{J}$,
  or if there is no such $c$ then set $c=0$; of all the intervals
  containing $c$ select one whose finish time is as late as
  possible and insert it into $J$.
  \end{subproblem}

\begin{solution}
    The algorithm is correct. This algorithm was implemented. 
\end{solution}

  \begin{subproblem}
    \textbf{Select Longest Interval:}
  Initialize $\mathcal{J}=\emptyset$.
  Until the intervals in $\mathcal{J}$ cover all of $[0,M]$,
  repeat the following loop: among all the intervals in
  $\mathcal{I}$ that are not already contained in the union
  of the intervals in $\mathcal{J}$, select one that is as
  long as possible (i.e., that maximizes $b_i-a_i$) and
  add it to $\mathcal{J}$.
  \end{subproblem}

\begin{solution}
$$
\mathcal{I} = \{[0_0,8_0], [8_1,9_1], [9_2,10_2], [0_3,5_3], [5_4,10_4]\}, [0,10]
$$
$$
\mathcal{J}_{ans} = \{[0_0,8_0], [8_1,9_1], [9_2,10_2]\}
$$
$$
\mathcal{J}_{opt} = \{[0_3,5_3], [5_4,10_4]\}
$$
\end{solution}

  \begin{subproblem}
    \textbf{Delete Earliest Redundant Interval:}
  Initialize $\mathcal{J}=\mathcal{I}$. While there
  exists a redundant interval in $\mathcal{J}$ --- one
  that is contained in the union of the other intervals
  in $\mathcal{J}$ --- find the redundant interval in
  $\mathcal{J}$ with the earliest finish time and
  remove it from $\mathcal{J}$.
\end{subproblem}

\begin{solution}
$$
\mathcal{I} = \{[0_0,8_0], [8_1,9_1], [9_2,10_2], [0_3,5_3], [5_4,10_4]\}, [0,10]
$$
$$
\mathcal{J}_{ans} = \{[0_0,8_0], [8_1,9_1], [9_2,10_2]\}
$$
$$
\mathcal{J}_{opt} = \{[0_3,5_3], [5_4,10_4]\}
$$
\end{solution}
    
\begin{subproblem}
    \textbf{Delete Latest Redundant Interval:}
  Initialize $\mathcal{J}=\mathcal{I}$. While there
  exists a redundant interval in $\mathcal{J}$ --- one
  that is contained in the union of the other intervals
  in $\mathcal{J}$ --- find the redundant interval in
  $\mathcal{J}$ with the latest finish time and remove it
  from $\mathcal{J}$.
\end{subproblem}

\begin{solution}
This answer is not correct.

$$
\mathcal{I} = \{[0_0,7_0], [7_1,8_1], [8_2,9_2], [0_3,5_3], [5_4,10_4]\}, [0,10]
$$

$$
\mathcal{J}_{ans} = \{[0_0,7_0], [7_1,8_1], [8_2,9_2]\}
$$
$$
\mathcal{J}_{opt} = \{[0_3,5_3], [5_4,10_4]\}
$$
\end{solution}

{\bf A note about tie-breaking:}
When implementing any of the algorithms above one must
choose a tie-breaking rule. In other words, when more than
one interval meets the criterion defined in the algorithm
specification (such as the latest finishing redundant
interval) one must specify which of the eligible
intervals is chosen. In the context of this problem,
if you are asserting that an algorithm is correct,
it should mean that you believe the algorithm gives
the correct answer on every input instance
{\em no matter how the tie-breaking rule is implemented;}
that is what your proof of correctness should show.
When you assert an algorithm is incorrect,
for full credit you should supply an input instance
that leads to an incorrect output {\em no matter how
the tie-breaking rule is implemented.} However,
significant partial credit will be awarded for
providing an input instance that leads to an incorrect
output {\em for some choice of tie-breaking rule},
though not necessarily for every tie-breaking rule.

\end{problem}
\newpage
%%%%%%%%%%%%%%%%%%%%%%%%%%%%%%%%%%%%%%%%%%%%%%%%%%%%%%%%%%%%%%%%%%%%%%%%%%%%%%%%
% Problem 3

\newcommand{\rank}{\textnormal{rank}}
\newcommand{\Colon}{:}
\newcommand{\dotdot}{..}
\newcommand{\numinv}{\textrm{NI}}
\newcommand{\numlargeinv}{\textrm{NLI}}

\begin{problem}
  We consider how to count the number of \emph{large} inversions between two
  rankings. Consider a set~$N$ with size $n$. 
  A ranking is a 1-1 function $\rank\Colon N
  \rightarrow [1\dotdot n]$. In other words, $\rank$ maps each item in $N$ to a unique
  integer, and therefore represents an ordering of the $N$ items where lower
  integer values indicate higher rankings. 

  As described in the Kleinberg-Tardos textbook, Section~5.3, we can compare two rankings, $\rank_i$ and $\rank_j$, by counting the number of
  inversions. We label each $x \in N$ with $\rank_i(x)$, hence making $\rank_i = [1, \dots ,n]$. Then, we define $\rank_j$ in terms of $\rank_i$ and count the number of inversions in $\rank_j$. 

  Suppose now we want to consider large inversions, where we define a threshold
  $0 < \delta < n$ and let $\numlargeinv_{\delta,i,j}$ equal the number of pairs
  $x,y \in [1\dotdot n]$ such that $x < y$ and $\rank_j(x) > \rank_j(y) + \delta$. In other words, we only consider inversions large if their ranking differs by at
  least~$\delta$.
  \begin{subproblem}
    (\textit{6 points})
    Design and implement a $\Theta(n^2)$ algorithm that iterates over every pair
    of points and determines $\numlargeinv_{\delta,i,j}$ given two rankings
    $\rank_i$ and $\rank_j$ and a threshold $\delta$. Describe the algorithm in the write-up, and include your implementation in \texttt{problem\_3/p3\_a.py}.
  \end{subproblem}

\begin{solution}
Please answer here.
\end{solution}

  \begin{subproblem}
    (\textit{14 points})
    Design and implement a $\Theta(n\log n)$ algorithm to calculate
    $\numlargeinv_{\delta,i,j}$. Include your implementation in
    \texttt{problem\_3/p3\_b.py} and explain your algorithm in the write-up.
    Using your local experimental system perform
    an empirical performance analysis of both implementations (see homework
    instructions in the first page) and determine
    for what~$n$ does the asymptotically faster implementation dominates.
  \end{subproblem}

\begin{solution}
Please answer here.
\end{solution}

\end{problem}

\newpage
%%%%%%%%%%%%%%%%%%%%%%%%%%%%%%%%%%%%%%%%%%%%%%%%%%%%%%%%%%%%%%%%%%%%%%%%%%%%%%%%
% Problem 4


\begin{problem}
  Given a sequence of integers $a_1,\ldots,a_n$ we want to find the
  number of times the most frequent pairwise difference appears.
  Let $\delta_{i, j} = a_i - a_j$ for $i \ne j$ in the list. Suppose you know one of the \href{https://en.wikipedia.org/wiki/Mode_(statistics)}{modes} (i.e. the most frequent value) of the list of all the $\delta_{i, j}$'s is $\delta_{mode}$. We want to count for how many values $\delta_{i, j} = \delta_{mode}$. 

  \begin{subproblem}
    (\textit{8 points})
    Design and implement a $\Theta(n \log n)$ algorithm that given a sequence of integers $a_1,\ldots,a_n$ and $\delta_{mode}$ as described above, outputs the frequency of $\delta_{mode}$. Include your implementation in \texttt{problem\_4/p4\_a.py}.
  \end{subproblem}
  
\begin{solution}
Please answer here.
\end{solution}
  
  \begin{subproblem}
    (\textit{7 points})
    Design and implement a $\Theta(n)$ algorithm that given a sequence of integers $a_1,\ldots,a_n$ and $\delta_{mode}$ as described above, outputs the frequency of $\delta_{mode}$. Include your implementation in \texttt{problem\_4/p4\_b.py}.
  \end{subproblem}

\begin{solution}
Please answer here.
\end{solution}
  
  \begin{subproblem}
    (\textit{5 points})
    Provide a comparison of the empirical performance analysis of your algorithms from parts (a) and (b). There is a clear performance advantage of using the algorithm from part (b) over the one from part (a). Is there any trade-off? \textit{\textbf{Hint:} Consider the \href{https://en.wikipedia.org/wiki/Space_complexity}{space complexity} of the algorithms, which is the space an algorithm takes to execute.}
  \end{subproblem}

\begin{solution}
Please answer here.
\end{solution}
  
  Note that the mode can be negative. If $\delta_{i, j}$ is a mode of the list,
  then $-\delta_{i, j} = \delta_{j, i}$ will be a mode too. The input to the
  function will be one of these modes, which can be positive, negative, or 0.
\end{problem}
\newpage

%%%%%%%%%%%%%%%%%%%%%%%%%%%%%%%%%%%%%%%%%%%%%%%%%%%%%%%%%%%%%%%%%%%%%%%%%%%%%%%%
% Problem 5

\begin{problem}
  Eman baked some cookies for her friends, and asked Avital to help
  by distributing them among their friends. 
  Their group of friends is very close-knit, so they all live in a building 
  together, called ``The Building''.
  
  The architect of ``The Building'' designed 
  the property such that every floor is a very long corridor. To one side of 
  the corridor are all the apartments in that floor, with the distance between 
  every pair of consecutive doors being always the same.   
  The architect is not very fond of elevators, so to the other side of the 
  corridor, there is a pair of staircases in front of every apartment door,
  leading to all the floors in the building. 
  Apartments are labeled first by floor and then by index in the floor, hence 
  $apt_{(12,21)}$ is the 21st apartment in the 12 floor.

  Avital is not a big fan of walking, so as she delivers the cookies,
  she may give some extra cookies to her friends, and ask them to
  distribute them to some other friends. All the friends may subsequently
  divide their cookies and ask other friends to help with the deliveries too. 
  
  Avital wants to minimize everyone's total number of steps. Assume that the distance between an apartment $apt_{(f,i)}$ and $apt_{(f,i+1)}$ is the same as that between $apt_{(f,i)}$ and $apt_{(f+1,i)}$. 
  
  Assume the friends are distributed across $n$ apartments,
  where each apartment, including Eman's, has $p > n$ friends
  currently hanging out.
  Finally, assume that Avital starts at apartment $apt_{(1,1)}$,
  where Eman lives, and she can recruit some of the friends currently there.

  (\textit{20 points}) Design and implement an algorithm that given a list of apartments where the friends live, returns pairs $(a, b)$, where friends from apartment $a$ take the cookies to the friends from apartment $b$, that minimize the total number of steps. Your algorithm should run in $O(n^2 \log n)$.
  Describe your algorithm in the write-up and include your implementation in \texttt{problem\_5/p5.py}.

\begin{solution}
Please answer here.
\end{solution}
\end{problem}

\begin{challenge}
    \textbf{A closer look at Integer Multiplication}
    
    Given two $n$-bit integers (represented as a list of binary values, i.e. base 2), $x, y\in\{0,1\}^n$,
    implement three approaches to computing their product $x \cdot y$
    and perform an empirical performance analysis of the algorithms.

    (\textit{2 points}) The first approach will be a reference implementation. Compute the product by converting from their binary representation to an int, then multiplying. Please provide a brief description of your implementation here.
    
    \begin{solution}
        Please answer here.
    \end{solution}

    (\textit{5 points}) For the second approach,
    we would like you to implement the Karatsuba algorithm,
    described in section 5.5 of the Algorithm Design textbook.
    Please provide a brief description of your implementation here.

    \begin{solution}
        Please answer here.
    \end{solution}

    (\textit{10 points}) The final algorithm to implement is the Fast Fourier Transform (FFT), described in
    section 5.6 of the Algorithm Design textbook.
    Please provide a brief description of your implementation here.

    \begin{solution}
        Please answer here.
    \end{solution}


    (\textit{3 points}) Finally, provide an empirical performance analysis of the algorithms.
    Ensure that the analysis samples enough problem sizes (i.e. integer sizes $n$)
    such that the runtimes can be clearly differentiated.
    
    \begin{solution}
        Please answer here.
    \end{solution}

\end{challenge}

\end{document}
